\chapter{Programming in {MicroWorldsEX}}

\epigraph{\emph This chapter introduces the features of MicroWorldsEX. It then provides installation guide, tutorials and examples for using MicroWorldsEX to make educational materials.}{}

\newpage

\section{Why MicroWorldsEX?}

MicroWorldsEX is a logo-based programming language\footnote{\url{http://el.media.mit.edu/logo-foundation/logo/programming.html}}, which uses a turtle object that can be modified and animated into any shape and size. Due to its syntax-based interface, it provides larger flexibility than other programming platforms available for education. This gives it an edge while designing objects incorporating local knowledges that are easily understandable to children there. 

MicroWorldsEX can be used as a tool to understand various concepts such as,

\begin{enumerate}
 

\item Understanding directions with X and Y coordinates.
\item Calculating angles and making geometrical shapes using commands.
\item Making animations to understand the action and learn vocabulary.
\item Composing graphics-based story books with up to 50 pages.
\item Carrying out simulations for science project.
\item Editing and adding tunes (including your own voice) to create a personalized music.
\end{enumerate}
With these functionalities, children can be engaged in creating their own project, which would invoke learning benefits like innovation and creativity without any bounds.

It follows the constructivist model, which enables it to create basic to robust programs with gradual increase in complexity. Moreover, its specifically designed for children and teachers and it is supposed to make logical learning fun with graphics and animations.

The online help has examples but, unfortunately, most of the materials are complex. Therefore the tutorials in this chapter are focused towards improving basic skills, which is absolutely necessary for users without prior knowledge of computer programming.






\section{Installation and Interface}
MicroWorlds EX is available to download from their website\footnote{\url{http://www.microworlds.com/shop/product\_info.php?cPath=23\&products\_id=34}}. Both Windows and Macintosh versions are available and can be used without any internet connection. Various languages are provided in the Windows version, including Spanish, Italian and Russian. A single license costs USD 99, and a 6-user lab pack costs USD 449, which is relatively affordable. In addition, a free demo version, which has full functionalities except saving or printing a project, is also available for download\footnote{\url{http://www.microworlds.com/solutions/demo\_ex.html}}. This version, with functionalities similar to other available platforms, can be used for unlimited time for practice purposes.


Finally, setup of MicroWorldsEX in Mac OS X is straightforward once the installation file is downloaded. Just locate the setup file (mwexdemo\_mac.dmg) in Downloads folder, double click and follow the on-screen instructions. Once the installation is complete, MicroWorldsEX appears in the launchpad menu and the Applications folder. 

Interface of MicroWorldsEX is easy to navigate and user friendly. It has a simple layout that even a new user should not have difficulty with. The interface can be divided into four different parts (see Figure~\ref{fig:MainScreen}):

\begin{figure}
    \centering
    \includegraphics[width=0.8\textwidth]{figs/MainScreen.jpg}
    \caption{Interface of MicroWorldsEX home screen.}
\label{fig:MainScreen}
\end{figure}



\begin{enumerate}


\item Main Display (Red Box): Displays the content of design, animations and the output of commands.
\item Command Center (Blue Box): Input commands here, which will create action to turtle object in the Main Display.
\item Object Center (Yellow Box): Select drawing objects like characters, and add animations to them.
\item Help Window (Green Box): Quick help for beginners.

\end{enumerate}



\section{Tutorials and Examples}

Some tutorials are provided here to start programming in MicroWorldsEX. A much useful and longer media have also been prepared. For those, visit \url{http://collaborativedistancelearning.net/tutorials.html}. Some examples are clearly explained with every step, but latter ones are just presented as an information of what can be done in MicroWorldsEX. Visit the website for full video instructions on those examples too.

\subsection{English Vocabulary}
\begin{figure}
\centering
\includegraphics[scale=0.25]{figs/Vocabanimation.png}
\caption{A snapshot of animation for english vocabulary.}
\label{fig:Vocabulary Animation}
\end{figure}

English vocabulary can be learnt easily by creating actions as animations. For example, the animal object doing some action is accompanied by a verb word. To create the animation, follow the steps below. 

\begin{enumerate}


\item Go to the Object Center.
\item Go to the first tab of Paintings/Clipart.
\subitem Listings of background will appear.
\item Choose the one you prefer.

\item Drag the background to the Main Display.
\item Right click on the background and click Stamp Full Page.
\item For the characters, as shown in the Figure~\ref{fig:Vocabulary Animation}, 
\subitem A hatching turtle must be placed on the Main Display on top of the background.
\item Now, go to the second tab of Paintings/Clipart and choose another character.
\subitem To animate a character, make the hatching turtle wear it.
\item Right click the character and click Open Backpack.
\begin{itemize}
	\item Go to Rules section and input commands (On Click - fd 100 wait 10) 
\subitem This command moves the character forward by 100 units distance, when the mouse button is clicked.
\item (Continue - fd 100 wait 10)
\subitem To continue to do the same thing to create motion of character.
\end{itemize}
\item To add the vocabulary text, go to the Menu Bar and click on Object and then New Text Box.
\item A text box will appear on the Main Display. You can customize the text box as per your preference.
\item Repeat the process for other characters.
\item Once done, click the floppy disk button to save the project.


\end{enumerate}


\subsection{Understanding Coordinates}

This tutorial is an example of those that can help understand coordinates and directions (or motion). It describes how and where the object is moving when you input a certain command.

In Figure~\ref{fig:Coordinates xcor and ycor Animation}, vertical and horizontal orientation are shown. Based on changing coordinates, for the boats are moving in, say, x-direction with changes in location shown by changing xcor value.


\begin{figure}
\centering
\includegraphics[scale=0.25]{figs/coordinates.png}
\caption{A snapshot of animation for understanding coordinates.}
\label{fig:Coordinates xcor and ycor Animation}
\end{figure}

\begin{enumerate}
\item Follow the same process as the previous example in Section~3.3.1 for selecting the objects. In this case, water body and boats.
\item The boat objects, by default, have the same characteristics. For each of them, right click and select Open Backpack and change the animation. 
\subitem Select xcor to move a boat in horizontal direction.
\item Set the rules to Glide command, if you want to move the object in both horizontal and vertical direction.

\end{enumerate}


 
\subsection{Filled Geometric Shapes}

This is a tutorial to help with drawings. With this feature, the user does not need to be limited to clipart objects available in MicroWorldsEX only. Creating your own abject with animations can make the learning process attractive, creative, and motivating.
\begin{enumerate}
\item Go to the Object Center.
\item Choose Painting Tools.
\subitem many options for drawing with colors on patterns of any shape. 
\item For simplicity, choose the Uniform Shape tool for circle and square.
\item Add a color to make it attractive.
	\subitem See Figure~\ref{fig:Geometric Shapes Painting} for visualization.
\item These objects can be animated by giving commands as in previous examples.

\end{enumerate}

\begin{figure}
\centering
\includegraphics[scale=0.25]{figs/Shapes.png}
\caption{A snapshot of drawing color-filled geometric shapes.}
\label{fig:Geometric Shapes Painting}
\end{figure}

\newpage
\subsection{Adding sounds}

Sounds and music bring life to animations. Music can be added from built in sounds,  or customized using snippets available as shown in Figure~\ref{fig:Making your own Melody Music}. 

The process is quite simple. 

\begin{enumerate}
	\item Click on the piano shaped object.
	\item Select and save your melody with a given name.
\end{enumerate}
\begin{figure}
\centering
\includegraphics[scale=0.25]{figs/sound.png}
\caption{A snapshot of object to create customized sound.}
\label{fig:Making your own Melody Music}
\end{figure}

\subsection{Making Graphicsbook}

The Main Display can handle up to 50 pages in total, which should be enough to prepare a good graphics-based book for children. For example, one can make animated handbooks to learn Khmer language with icons and animations (see Figure~\ref{fig:Khmer Learning with page transition}). The book includes the “Go Back" and “Continue" buttons for easy navigation. There is also a command to add transition animation between pages.

\begin{figure}
\centering
\includegraphics[scale=0.25]{figs/khmer.png}
\caption{A snapshot of adding pages to create a graphics-based book with animation for transition.}
\label{fig:Khmer Learning with page transition}
\end{figure}

\subsection{Using Commands}

The most important feature of MicroWorldsEX is its ability to handle any commands. So, understanding the commands and how to make it work in the Main Display is very important to be fully able to use it. First, the user should always add the hatching turtle. Then every command input in Command Center is executed and visible via action of the hatching turtle. 

Always use square brackets to input commands as shown in Command Center of Figure~\ref{fig:ContinousCircle animation}.


\begin{figure}[h!]
\centering
\includegraphics[scale=0.25]{figs/ComplexShapes.png}
\caption{A snapshot of process to make numerous circles using commands.}
\label{fig:ContinousCircle animation}
\end{figure}

\newpage
Finally, some examples to make complicated animations using commands are below,
\begin{enumerate}
	\item To make a square:

\begin{lstlisting}
repeat 4 [forward 50 right 90 wait 10]
end
\end{lstlisting}

\item To make a circle:

\begin{lstlisting}
cg repeat [repeat 360[fd 200 * 3.14 / 360 rt 1]rt 40]
repeat 4 [forward 50 right 90 wait 10]
end
\end{lstlisting}

\item To make numerous circles as shown in Figure~\ref{fig:ContinousCircle animation}:

\begin{lstlisting}

cg repeat 10[repeat 360[fd 200 * 3.14 / 360 rt 1]rt 40]
repeat 4 [forward 50 right 90 wait 10]
end
\end{lstlisting}

\end{enumerate}
