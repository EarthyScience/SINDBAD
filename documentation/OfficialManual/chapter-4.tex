\chapter{Summary and Recommendation}
\epigraph{\emph The findings of the practicum are summarized and recommendations are made.}{}


\newpage

In collaboration with a Denver-based INGO, the Children’s Future Organization, this practicum evaluated the education methods and resources that would be suitable to support technology-based education for a school in rural Cambodia.

The review of existing studies and education system in Cambodia suggests that the system follows traditional method and the subject matters in the primary and secondary level mostly cover the local language Khmer, Mathematics, and Science only. Moreover, information and communication technology related courses are not offered until the 11$^\mathrm{th}$ year of education, which highlights the need and difficulty in enabling technology-based education there.

Various resources that are used in technology-based education were then reviewed. A short summary of the main features of those resources is presented in Table~\ref{tabsum}. The resources are either available as websites or as programs, while some of them allow for creating an object and then a teaching material. 




\begin{table}[bh!]
    \caption{Summary of the major Technology-based educational resources}
     \begin{tabular}{ c | C{3.1cm} | L{4.8cm} | L{4.5cm} }
        \toprule
        SN & Provider & Content & Platform  \\
        \midrule
 1 & JumpStart\footnote{\url{http://www.jumpstart.com}} &  Animations, educational games, notes & Mobile app and website, worksheet   \\ \hline
 2 & BBC Schools\footnote{\url{http://www.bbc.co.uk/schools/0/}} & Advanced graphics with animations &   Website, primary and secondary education \\ \hline
 3 & Scratch\footnote{\url{https://scratch.mit.edu}} & Basic graphics animations, and also create & Website, web-based application, community  \\ \hline
 4 & Alice\footnote{\url{http://www.microworlds.com/index.html}}     & Basic graphics with animations & Website, software \\ \hline
 5 & MicroWorldsEX\footnote{\url{http://www.alice.org/index.php}}  & Basic to advanced graphics with animations & Website, software \\ 
        \bottomrule
     \end{tabular}
  \label{tabsum}
\end{table}

\newpage
Then, the following criteria was used to select the most viable option.

\begin{enumerate}
	\item Availability for offline use: This is an important factor because only 2 out of 1000 Cambodians have broadband internet connection\footnote{\url{https://www.itu.int/ITU-D/asp/CMS/Events/2011/ITU-ADB/Cambodia/Telecom_Infrastructure_MPTC.pdf}}.
	\item Flexibility in creating objects: To be able to create a locally relevant education materials.
	
	\item Long-term suitability.
	
\end{enumerate}

Based on the criteria, JumpStart and BBC Schools were not selected as they are online only platforms without a feature to create. Scratch only has a web-based software, and is not available offline. Alice is available offline but is based on 3D-objects which has lower flexibility compared to logo programming syntax-based MicroWorldsEX. Therefore, MicroWorldsEX was chosen, and it also includes functionalities of other platforms. As it is designed for secondary level education as well, it should be applicable in the longer-run also. It should, however, be noted here that other resources and platforms are not necessarily inferior. For example, mobile educational games from both JumpStart and BBC should be very helpful as an extra educational material. If possible, it is recommended to use multiple resources as they have their own benefits.


Finally, numerous simple and easy-to-understand tutorials and videos on installation and use of MicroWorldsEX have been created. These tutorials are either available in this report or a newly-created knowledge-sharing website at {\small \url{http://collaborativedistancelearning.net}}.


Considering the current education system in Cambodia, which focuses on multiple subjects with at least 20 hours of lecture a week, a blended learning in addition to current school might be the most viable method. Previous experiences of such platform, such as those from Open World Learning schools, showed improvements in both learning and enthusiasm of participating students.

