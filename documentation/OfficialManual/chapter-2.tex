\chapter{Education Methods and Resources}
\epigraph{\emph This chapter provides discussion on different education methods. It then provides a short description of the most relevant technology-based educational resources and products that are commonly used. The aim is to provide information about various alternatives of education methods and resources.}{}

\newpage 

Education method changes with time to meet the new demand and use the resources. Broadly speaking, traditional education refers to long-established methods of teaching, and modern education can refer to anything that is deviating from traditional ones. Over time, education reform has focused in a more holistic approach to satisfy individual students' needs and self-expression. This chapter first focuses on different education methods and their features, and then reviews the most relevant resources to enable technology-based education.



\section{Education Methods}

Over the last few decades, education has changed significantly in 4 key areas: teacher and student roles, curriculum, assessment, and pedagogy\footnote{\url{http://www.arcade-project.org/21st-century-education-vs-traditional-education/}}. Teachers are source of knowledge and more oriented towards improving the learning process. Students are more actively engaged and responsible for their own learning. Mistakes are considered as parts of the learning process. Instructions are flexible to meet the level of students to maximize the potential. In the following, features of some traditional/conventional, modern , and blended education methods are provided.


\subsection[Conventional/Traditional]{Conventional/Traditional Method}

Here are some methods that are commonly used in conventional education.
\begin{enumerate}

\item Writing center, book area, puzzles and manipulations

\begin{itemize}
\item Memorizing the alphabets and then words.

\item  Repeatedly writing sentences to get a better understanding.

\item Puzzles and manipulations to get a concept of alphabets and meanings.


\end{itemize}



\item Small Groups

\begin{itemize}
\item Assigning children to small group activities to introduce the concept of “sharing is learning." 

\item Simple tasks such as telling stories, expressing the activities, and interacting.

\item Simple tasks to learn mathematics for e.g., building blocks, lego.

\item Using pictures to help children memorize.

\end{itemize}

\item Conversations

\begin{itemize}
\item Conversations about class topics with formal or informal sets of language. It helps develop children's social  and communication skills.


\end{itemize}

\item Weekly Assignments
\begin{itemize}
\item Assignments are targeted to make students independent. 
\item The teachers evaluate students' ability to finish their jobs.
\item Students develop ability to synchronize the knowledge with gaps in solving problems.
\item Development of framework for solving problems.


\end{itemize}

\item Monthly Tests
\begin{itemize}
\item Stricter than assignments.
\item Check the performance level under pressure. 
\item Prepare students to perform better under any circumstances.
\item Grade the students to identify the weaker ones and increase focus on them through engagements.

\end{itemize}
\end{enumerate}

\subsection[Modern]{Modern Method}

The modern education focuses on learning beyond books and school boundaries. They can often be technology oriented and more engaging.


\begin{enumerate}



\item Logical learning techniques
\begin{itemize}
\item Correct way of learning is very important to understand and remember things. 

\item Differentiate between logic, concept and cognitive memory, and prepare individual and group tasks. 

\item An example of remembering using logic. 

\begin{itemize}
\item Not many can easily remember the following 13 alphabets. 

\begin{lstlisting}
	   U R H A Y T H S A Y P D P
\end{lstlisting}

\item In a logical way, its just two words. 

\begin{lstlisting}
	   H A P P Y T H U R S D A Y  
\end{lstlisting}

\end{itemize}
	
   \item “Concept is forever and fact is temporary." -  TED Talk by Professor Dr. Marty Lobdell from Pierce College\footnote{\url{https://www.youtube.com/watch?v=IlU-zDU6aQ0}}.
\end{itemize}

\item Story time and family involvement with media
\begin{itemize}
\item Development of thoughts progresses the ability of thinking.
\item Advance the initial thinking level towards active implementation. 
\item Develop speaking by listening.

\end{itemize}

\item Songs, wordplay and letters (educational games in smartphones and tablets)
\begin{itemize}
\item Develops phonological awareness.
\item Alphabet-word connection.
\item Learning vocabulary with fun literacy circle activities.

\end{itemize}

\item Using computer programming

\begin{itemize}
\item Programming is the process and concept of logic, which is implemented via code (coding) to solve the problem.

\item Multiple platforms to improve children's logical thinking (see next section).
\item Improve creativity, for e.g., by creating music using audio.
\item Writing picture stories with graphics to improve imagination/exploration and thinking skills.
\item Drawing geometrical shapes to understand concepts.

\end{itemize}

\item Technical summer camps

\begin{itemize}
\item Same age children groups in a camp to learn technical skills (e.g., LEGO Robotics, STEM challenges of Sphere Orbotix technology\footnote{\url{http://www.gosphero.com/education/}}, Digital Maze, Game Design by MicroWorlds).
\item Follows “play is a powerful teacher." (SPRK\footnote{\url{https://www.sparkplay.com}}). 
\item Provides platform for students to learn the concepts of advanced topics like programming, robotics, and math in a fun way.


\end{itemize}

\end{enumerate}

\subsection[Blended]{Blended Method}

As both conventional and modern methods have their own merits and demerits, there are some methods which attempt to combine the positives from both. These are known as blended learning methods. In blended learning method, 

\begin{itemize}
	\item The students use both modern and traditional methods to learn. 
	\item Parts of the courses might be taught in class while part at home. 
	\item More flexible on contents and time management. 
	\item More accessible and organized.
\end{itemize}

The most well known example of blended learning is the Open World Learning (OWL)\footnote{\url{http://www.openworldlearning.org}}. It is a after-school program that encourages students to pursue their interests at their own pace. It provides opportunity  to early STEM education with an ultimate aim to increase the number of STEM graduates. Students participate in online presentations and mini-competitions, which encourage learning and use of technology.
It also increases pedagogy through discussion, exploration and investigation, mostly in digital literacy as the students spend much more time in computer than they do in their day schools. The constructivist teaching materials are designed to encourage curiosity  and responsibility through self-reporting. Studies have shown that students participating in OWL programs are performing better academically. 

\section{Resources for Technology-based Education}

This section briefly shows the most important features of commonly used platforms and programs used in technology-based education method. 
 

 
 
\subsection{JumpStart}
Since 1991, JumpStart (also known as Knowledge Adventure) has been developing educational games and animations with an aim of making learning fun\footnote{\url{http://www.jumpstart.com}}.

It has a large user community with 30 million members, and collaborates with major animation companies to create educational materials. Most of the materials are provided in the online gaming platform as 3-D animations and games.

The games are for topics that are supplemental to what is taught in schools, thereby promoting a blended learning environment. It also provides free games for both iOS and Android mobile devices. The downloadable materials and worksheets are available for both parents and children.

\subsection{BBC Schools}
Going by the name Bitesize\footnote{\url{http://www.bbc.co.uk/schools/0/}}, the comprehensive website  has all the BBC’s formal education content for most of the subjects taught in primary and secondary education from across United Kingdom in form of computer graphics.

It covers a wide range of topics from Arts and Science to Computing and should be an effective tool (video with transcript) to introduce concepts. A fraction of topics, especially for Mathematics, is also presented as games to promote learning as fun.

As the target users are children of United Kingdom, where 84\% of the household have internet connection, the materials are provided online. This limits the applicability in places where Internet connections are not available or less reliable.

\subsection{Scratch}
While the above two platforms provide a plethora of information that are useful for education, they don't allow creating materials. 

In 2003, MIT launched an online platform called Scratch\footnote{\url{https://scratch.mit.edu}} to help improve digital fluency related to creating, designing, and mixing objects. With Scratch, interactive stories, games, and animations can be created and shared in an online community. Scratch aims to help children learn how to think creatively on a collaborative work.

Scratch has a large online community with over 6 million users from 150 countries around the world. It is available in 40 different languages.
 
 \subsection{MicroWorlds}
Founded in 1981, MicroWorlds is the programming platform developed by LCSI\footnote{\url{http://www.microworlds.com/index.html}}, which is a publisher of constructivist educational softwares targeted at K-12 schools around the world. 

MicroWorlds environment encourages students to explore and test their ideas through science, mathematics, interactive multimedia. It mainly aims at improving children's creativity.

It is a syntax-based platform. So, despite being complicated than other graphical drag-and-drop platforms discussed before, it provides much larger flexibility and is also useful for adults and teachers to create their own courses. Both versions of MicroWorlds, Junior (free) and EX (paid) can be installed as a program in Windows and Macintosh computers for offline use.

\subsection{Alice}
The Alice Project is a multi-university initiative\footnote{\url{http://www.alice.org/index.php}}, with an aim of teaching computer programming using 3-D objects targeted at improving the imagination of children.

Alice is suitable to create contents and animations for telling a story, playing an interactive game, or creating a video. It allows learner to learn fundamental concept of object-oriented programming. Alice is freely available for all operating systems, and has good platforms for creating community with services such as mailing lists, social networking and blogs.

Use of Alice has shown to not only improve the programming skills of the students, but also change their attitude towards computer programming\footnote{\url{http://www.alice.org/publications/EvaluatingTheEffectivenessOfANewApproach.pdf}}.






