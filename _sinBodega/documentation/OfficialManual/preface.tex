\chapter*{Abstract}
\setheader{Preface}

This practicum focused on evaluating the methods and resources that can be used to prepare technology-based education materials for a school in rural Cambodia. It was carried out as a joint effort of the ATLAS Institute, University of Colorado Boulder and a Denver-based INGO, the Children's Future Organization. This report introduces education methods, investigates available resources, and finally presents tutorials describing the use of resources suitable for primary level education. 



Even though there are various resources and platforms available, MicroWorldsEX, a LCSI product, is recommended, as it can be used offline and is based on logo programming language. Due to its syntax-based interface, it provides larger flexibility in producing teaching materials oriented towards local information and culture. Various simple and easy-to-understand tutorials for MicroWorldsEX were prepared, and a website ({\small \url{http://collaborativedistancelearning.net}}) has been created to share the information and knowledge.


\begin{flushright}



{\makeatletter\itshape
    \@author \\
\href{mailto:milee.shrestha@colorado.edu}{milee.shrestha@colorado.edu}\\
    Boulder, Colorado
	
	May, 2015
\makeatother}
\end{flushright}

